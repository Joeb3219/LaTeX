\documentclass[12]{article}
\usepackage[utf8]{inputenc}
\usepackage{blindtext}
\usepackage{amsmath}
\usepackage{fullpage}
\usepackage{setspace}
\usepackage [english]{babel}
\usepackage [autostyle, english = american]{csquotes}
\MakeOuterQuote{"}

\setlength{\parindent}{0em}
\setlength{\parskip}{1em}

\title{Evaluating the Responsibility of the Programmer for His AI}
\author{Joseph A. Boyle}
\date{December 20, 2017}

\begin{document}

\maketitle

\section{Introduction}
Sine the days of Alan Turing\cite{turing_test}, the question of whether or not machines can truly think has troubled even the brightest computer scientists, philosophers, and cognitive scientists. Moreover, the consequences of that question carry heavy consequences which have made the discussion space difficult to prove one way or another. If a robot commits a war crime, does its ability to think and rationally decide to commit that crime leave it liable? If so, does the lack of a real thought process delegate responsibility elsewhere? If the penalty of a crime is death, and the robot isn't ultimately responsible, does its creator gain the death penalty?

\section{Classes of Robots}
For the purposes of our discussion, the definition of a "bot" is important. 
	
	\subsection{Non-Mobile Bots}
		"Non mobile" bots are the types of AI that have been developed for the predominant duration of AI research. These types of bots include chatbots, infobots, weather-predicting bots, and news bots. Bots of this class typically run 
	
	\subsection{Mobile "Real Life" Bots}

	\subsection{Bots capable of Reproducing}

\section{Consequences}
	In the same way that most doctors would likely not lobby in favour of harsher punishments for mistakes, it is scary as a programmer to recommend programmers be held responsible for the use of their inventions. There are a number of consequences that should be considered with this view, and perhaps even circumstances where we hold programmers to be not responsible for what has happened with their inventions. That is, in the same way that doctors are held responsible for mistakes in a subset of cases, programmers should not be held liable in all situations.
	
	

\section{Conclusion}

\newpage

\begin{thebibliography}{6}

\bibitem{turing_test}
https://content.sakai.rutgers.edu/access/content/group/ffe89ccf-3ee0-4996-8775-5ac9e5fff069/Assigned\%20Readings/Turing\%20Computing\%20Machinery\%20and\%20Intelligence-1.pdf

\end{thebibliography}

\end{document}